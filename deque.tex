Двунаправленная очередь - очередь, в которой элементы можно добавлять как в конец, так и в начало.

Устройство элементов:
Двунапраленная очередь может быть реализована с помощью двусвязного списка
Двусвязный список состоит из узлов, каждый из которых содержит данные, а также
указатель на предыдущий и следующий элемент.

Push/Pop

PushFront - добавление в начало;
указатель на следущий элемент переднего элемента указывает на добавленный элемент,
а указатель на предыдущий элемент добавленного элемента указывает на него.
Указатель на следующий элемент нового элемента пуст;

PushBack - добавление в конец;
указатель на предыдущий элемент последнего элемента указывает на добавленный элемент;
указатель на следующий элемент добавленного элемента, в свою очередь, указывает на него.
Указатель не предыдущий элемент пуст;

PopFront - удаление из начала;
передний узел (данные и оба указателя) удаляется;

PopBack - удаление с конца;
последний узел (данные и оба указателя) удаляется;

Оценка сложности

Добавление/Удаление - O(1) - в худшем и среднем случае
Поиск элемента - O(N) - в худшем и среднем случае
Индексация - O(N) - в худшем и среднем случае

